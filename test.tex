\documentclass{article}

\usepackage
    { fontspec
    }

\usepackage{othercoder}[withcommands]

\begin{document}

\begin{othercoder}
//> |
//> ~
    #include <vector>
    #include <algorithm>

//> !
    class value_gte {
    public:
        value_gte(int comp)  // | An inline comment that spans multiple lines
            : m_comp(comp)   // | and explains the `value_gte` constructor
        {
        }

        template <typename ValType>                         // | <1>
//>           ^^^^^^^^^^^^^^^^                              // |
        bool operator() (ValType &&value) const {           // |
            return std::forward<ValType>(value) >= m_comp;  // |
//>            ^^^^^^^^^^^^^^^^^^^^^^^^^^^^
        }                                                   // |

    private:
        int m_comp; // <2>
    };

//> ~
    // The less important part of the example
    int main(int argc, char *argv[])
    {
        value_gte gte42(42);
        std::vector<float> xs { 1, 2, 4, 8, 16, 32, 64, 128, 256 };
        std::cout << std::count_if(std::cbegin(xs), std::cend(xs), gte42)
                  << std::endl;
    }
\end{othercoder}

Some in-text explanation of \othercoderCircled{1},
which marks a block in the above listing,
and a specific line \othercoderCircled{2}
which is just a line and nothing more.

\end{document}
